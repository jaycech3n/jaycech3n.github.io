\documentclass[12pt,a4paper]{article}

\usepackage[inner=3.3cm, outer=3.3cm, top=3cm, bottom=3cm]{geometry}
\usepackage[indent=3ex]{parskip}
\renewcommand{\baselinestretch}{1.06}

\usepackage{microtype}

\usepackage[T1]{fontenc}
\usepackage[p,osf]{cochineal}
\usepackage[cochineal,vvarbb]{newtxmath}
\usepackage[bb=libus,cal=boondoxupr,frak=euler]{mathalpha}
\usepackage[scale=0.9]{newtxtt}
\usepackage{nunito}

% \usepackage{libertine}
% \usepackage{palatino}
% \usepackage[osf,sups]{Baskervaldx}

\usepackage[dvipsnames]{xcolor}
\usepackage[colorlinks,urlcolor=NavyBlue]{hyperref}
\usepackage{graphicx}
\usepackage{enumitem}
\setlist{leftmargin=4ex, labelsep=2ex}

\newcommand{\hpar}[1]{\noindent\textbf{#1.}}

\begin{document}

\noindent Dear Symbolica AI Team, \\[2ex]
\phantom{x}\hspace{2ex}I am a mathematician and computer scientist specializing in category theory, type theory and interactive theorem proving, with experience working with machine learning in both academic and industry settings.
% and proof assistants.
I am currently completing my PhD
%on the construction of categorical structures in type theory
at the Functional Programming Lab of the University of Nottingham, and am due to submit my thesis in September this year.
Thus I am keen to apply for the position of Category Theory Scientist at Symbolica in order to use category theory to structure our understanding, design and implementation of machine learning models.
%, just as we have successfully done in the field of functional programming languages.
\\[3ex]
%
\hpar{Category and type theory research}
% The goal of my research is to
% % understand how to
% use type theory to express, reason about, and compute with higher
% % and \(\infty\)-
% categorical structures that arise in mathematics, software, and physical systems.
% My current category theory research is on approximate constructions of semisimplicial types and more general Reedy fibrant inverse diagrams in homotopy type theory, higher models of type theory, and coherences for such constructions.
% While my current PhD research
% % (on internal models of HoTT and the coherence problem of semisimplicial types)
% is technically in a different area than that of the published categorical deep learning literature, many of the underlying considerations and required skills are the same.
% % \footnote{And of course, I am familiar with monad algebras, monoidal categories, and other basics.}
% In particular,
My current PhD research is on the coherence problem of semisimplicial types.
While seemingly in a different area to that of the published categorical deep learning literature, it heavily involves simultaneously developing versions of the same categorical structures at wildly different abstraction levels.
For instance, one major goal of my research is to show how key structures in simplicial models of \((\infty, 1)\)-categories
% matching objects of Reedy fibrant diagrams %
% \footnote{Key structures in simplicial models of \((\infty, 1)\)-categories.}
may be defined in homotopy type theory.
% via opfibrations of linear cosieves over inverse categories.
This involves going back and forth between developing the necessary abstractions on paper, and using the Agda programming language to implement more direct and efficient “lower-level” encodings, which have to balance economy of definition and ease of use while still ensuring correctness.
% enforcing correctness (i.e.\ that the implementation provably conforms to the abstraction).

I also collaborate with my research group on various topics in type theoretic category theory, and this has so far resulted in a \href{https://types2023.webs.upv.es/slides/S10/TYPES2023-Chen-deJong-Kraus-Pradal.pdf}{formalized and refereed presentation} at the TYPES conference last year.
My previous research also includes an \href{https://drops.dagstuhl.de/entities/document/10.4230/LIPIcs.ITP.2021.12}{ITP 2021 conference paper} on the implementation of homotopy type theory in the Isabelle proof assistant, and a multiply cited \href{https://scholar.google.com/citations?view_op=view_citation&hl=en&user=OOY9aVYAAAAJ&citation_for_view=OOY9aVYAAAAJ:2osOgNQ5qMEC}{Honours thesis} on diagrammatic categorical quantum algebra. \\[3ex]
%
\hpar{Machine learning experience}
I have had a longstanding interest in machine learning systems from since before the current boom.
Before my PhD, I was a researcher
%a researcher at the Fraunhofer Institute in Germany
developing and implementing probabilistic graph models for natural language processing in civil service/industry, and briefly
% at the University of Innsbruck
on ML-assisted ``hammer'' tools for interactive theorem provers.
I was also a graduate teaching assistant for Masters-level machine learning classes at the University of Bonn, leading tutorial classes on neural networks, knowledge discovery and extraction, and data mining.

As deep reinforcement learning became more popular and ad hoc models proliferated, I pivoted (back) to category theory and type theory to better understand the fundamental structures underlying our logical and computational systems.
So it is quite exciting now to read the categorical learning literature, and get a feeling of its potential to both explain existing models and guide the development of new ones. \\[3ex]
%
\hpar{Communication in research}
I believe that good mathematical research rests not just on the basis of solid rigor but also good mathematical \emph{intuition}, and hence that one of the purposes of scientific communication in our field should be to transmit intuition effectively.
To this end, I strive to make my research talks as clear and understandable as possible, as evidenced for example by my \href{https://www.youtube.com/watch?v=fGnIdt_jPfA&t=4630s}{conference talk} at ITP 2021, \href{https://conferences.leeds.ac.uk/yamcats/meeting-29/}{invited talk} at the 29th YaMCATS meeting and my \href{https://www.youtube.com/watch?v=cB9hml8YBj4}{talk at the recent EuroProofNet WG6 meeting}.

It has also been my experience that informal discussion with other researchers plays an equally vital and effective role in developing mathematical intuition, and so I do my best to encourage this where I can.
In Nottingham I frequently have research discussions with other group members on topics in category theory, type theory and functional programming, and have also organized PhD seminars and a \textit{Sheaves in Geometry and Logic} reading group.
I was also the organizer of the main group seminar for three years, organizing talk programs and social activities for group members and invited speakers (and starting a Christmas tradition of playing Among Us for the final seminar meeting of the year). \\[3ex]
%
% As a scientist, I care deeply about how we can guide our technologies towards the freedom and flourishing of our societies to the best of our ability.
% This means that I care a lot about the purposes to which we put our machine learning systems, and the responsible democratization of their control and use.
% I believe that a good societal understanding of the capabilities and limitations of the systems that we create is a crucial part of the process, and that as scientists it is our responsibility to develop this understanding mathematically in a rigorous way, and to communicate it broadly and accessibly.
% To the extent that categorical deep learning at Symbolica holds the potential to further this ideal, I would like to be able to join the effort.
%
\noindent Thank you very much for your consideration.
I look forward to hearing back.\\

\noindent With best regards,\\
\hspace{-4ex}\includegraphics{signature}

\hspace{-2.5ex}Joshua (Josh) Chen

\end{document}