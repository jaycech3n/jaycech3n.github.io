\documentclass[11pt]{article}

\usepackage[inner=3.3cm, outer=3.3cm, top=3cm, bottom=3cm]{geometry}
\usepackage[indent=2ex,skip=0.5\baselineskip plus 2pt minus 2pt]{parskip}
\renewcommand{\baselinestretch}{1.1}

\usepackage{microtype}
% \usepackage{libertine}
\usepackage[osf,sups]{Baskervaldx}

\usepackage{enumitem}
\setlist{leftmargin=4ex, labelsep=2ex}

\usepackage{pifont}

\begin{document}

\noindent Dear Symbolica AI team,

I am a mathematician and computer scientist specializing in category theory and type theory.
% and proof assistants.
I am currently completing my PhD
%on the construction of categorical structures in type theory
at the Functional Programming Laboratory of the University of Nottingham, and am due to submit my thesis in September this year.

The goal of my research is to
% understand how to
use type theory to express, reason about, and compute with higher
% and \(\infty\)-
categorical structures that arise in mathematics, software, and physical systems. I am thus keen to be considered for the position of category theory scientist at Symbolica, to apply category theory and type theory to structure our understanding, design and implementation of machine learning models, just as we have successfully done for the field of functional programming languages.

My current category theory research is on constructing semisimplicial types in homotopical type theory, higher models of type theory, and coherences for such constructions.
% My current category theory research is on approximate constructions of semisimplicial types and more general Reedy fibrant inverse diagrams in homotopy type theory, higher models of type theory, and coherences for such constructions.
While this not exactly in the same niche as the theory in the published categorical deep learning literature, many of the underlying considerations are the same.
% \footnote{And of course, I am familiar with monad algebras, monoidal categories, and other basics.}
In particular, my work heavily involves simultaneously developing versions of the same categorical structures at wildly different abstraction levels.
For instance, one of the main constructions in my thesis is that of matching objects of Reedy fibrant diagrams, using opfibrations of linear cosieves over inverse categories.
Eliding the question of what these abstractions are, one major goal of my work is to show how they may be concretely encoded in type theory.
I do this in the Agda language, implementing more direct and efficient ``lower-level'' encodings that have to simultaneously balance between economy of definition and ease of use, while still enforcing correctness (i.e.\ that the encoding provably conforms to the abstraction).

% My previous research on diagrammatic categories for quantum topological invariants has also...

I have also had a longstanding interest in machine learning systems from since before the current boom.
Before I began my PhD, I was a researcher at the University of Innsbruck working on ML-assisted tools for interactive theorem proving, at the Fraunhofer Institute in Germany developing and implementing probabilistic graph models for natural language processing in industry, and a graduate teaching assistant for Masters-level machine learning classes at the University of Bonn, leading tutorial classes on neural networks, knowledge discovery and extraction, and data mining.
As the deep reinforcement learning boom hit and ad hoc models proliferated, I pivoted (back) to category theory and type theory in order to be able to better understand the fundamental structures underlying our logical and computational systems.
So it is quite exciting now to read the categorical learning literature, and get a feeling of its potential to explain the models we have today and guide the development of new ones.

In addition to my great technical interest, I care deeply about guiding our technologies towards the freedom and flourishing of our societies to the best of our ability. This means that I care a lot about the purposes to which we put our machine learning systems, and the responsible democratization of their control and use. I believe that a good societal understanding of the capabilities and limitations of the systems that we create is a crucial part of the process, and that as computer scientists it is our responsibility to develop this understanding mathematically in a rigorous way, and to communicate it broadly and accessibly. To the extent that categorical deep learning at Symbolica holds the potential to further this ideal, I would like to be able to join the effort.

\noindent Thank you very much for your consideration,

Josh Chen

\end{document}